\documentclass[12pt, margin=.5in]{article}

\usepackage{fontspec}
\usepackage{hyperref}
\usepackage{natbib}
\usepackage{amsmath}

\defaultfontfeatures{Mapping=tex-text}
\setmainfont {Adobe Garamond Pro} % Main document font
\setsansfont {Gill Sans} % Used in the from address line above the to address

\let\inlinecode\texttt

\pagestyle{empty}

\begin{document}
\vspace*{-6em}
\begin{center}
{\Large ECON 8320\   \ -- \ Tools for Data Analysis \\[.5em] Assignment 6 [25 points]
}
\end{center}

\setlength{\unitlength}{1in}

\hspace*{-4em}\begin{picture}(6,.1) 
\put(0,0) {\line(1,0){6.25}}         
\end{picture}
\hspace*{2em}
 
\begin{large}
Many of the functions that we perform in Python can be executed in parallel for increased performance. Let's start by re-writing one of the algorithms from Homework 2 using the \inlinecode{multiprocessing} library. The functions written in this assignment should check the number of available processor cores, and divide work evenly between the available cores. Parallelize the follwing functions:

\begin{enumerate}
\item Find the largest (or smallest) number in a list of any size
\item Use the csv module (documentation at \url{https://docs.python.org/3/library/csv.html#module-csv}) to read a csv file, and compute summary statistics for the third column. Summary statistics that should be calculated are minimum, maximum, mean, and standard deviation.
\end{enumerate}

\vfill Note: If you wish to parallelize the computation of mean and standard deviation (as you would for a large dataset), this blog post explains how to update means and standard deviations when incorporating a new batch of data: \url{http://notmatthancock.github.io/2017/03/23/simple-batch-stat-updates.html}
\end{large}


\end{document}